\usepackage[spanish,es-noshorthands]{babel}
\usepackage{tikz, pgfplots, geometry, graphicx, wrapfig, tipa, circuitikz} % Entornos graficos
\usepackage{amssymb, amsmath, float, mathpazo, textcomp, gensymb, mathtools} % Matematica/Simbolos
\usepackage{multicol, multirow, xcolor, colortbl}
\usepackage{lastpage, bookmark, authblk}
\usepackage{subcaption}
\usepackage{array}

% Varios
\usepackage[label=corner]{karnaugh-map}
\usepackage{threeparttable}
\usepackage{ulem, lipsum}
\usepackage{adjustbox}
\usepackage{listings}
\usepackage{fancyhdr}
\usepackage{capt-of}
\usepackage{caption}
\usepackage{cancel}
\usepackage{makecell} % Para hacer saltos de linea en tablas
\usepackage[utf8]{inputenc} % Required for inputting international characters
\usepackage[T1]{fontenc} % Output font encoding for international characters

% Librerias extra de Tikz
\pgfplotsset{compat=1.7} % !!VERSION PGF¡¡
\usetikzlibrary{patterns}
\usetikzlibrary{positioning}
\usetikzlibrary{arrows}
\usetikzlibrary{calc}
\usetikzlibrary{fpu}
\usepackage{longtable}

\definecolor{vgreen}{RGB}{104,180,104}
\definecolor{vblue}{RGB}{49,49,255}
\definecolor{vorange}{RGB}{255,143,102}
\definecolor{codegreen}{rgb}{0,0.6,0}
\definecolor{codegray}{rgb}{0.5,0.5,0.5}
\definecolor{codepurple}{rgb}{0.58,0,0.82}
\definecolor{backcolour}{rgb}{0.95,0.95,0.92}
\definecolor{mlgb}{RGB}{100,224,238}
\definecolor{mdrd}{RGB}{160,24,82}
\definecolor{mdblu}{RGB}{20,91,112}
\definecolor{mblk}{RGB}{35,35,35}





%Caratula
\newcommand{\subtitledoc}[1]{\newcommand{\@subtitledoc}{#1}}
\newcommand{\instituto}[1]{\newcommand{\@instituto}{#1}}
\newcommand{\carrera}[1]{\newcommand{\@carrera}{#1}}
\newcommand{\professor}[1]{\newcommand{\@professor}{#1}}
\newcommand{\catedra}[1]{\newcommand{\@catedra}{#1}}
\newcommand{\curso}[1]{\newcommand{\@curso}{#1}}
\newcommand{\legajo}[1]{\newcommand{\@legajo}{#1}}

%Encabezado/footer
\newcommand{\footerauthor}[1]{\newcommand{\@footerauthor}{#1}}
\newcommand{\footerlegajo}[1]{\newcommand{\@footerlegajo}{#1}}
\newcommand{\footercatedra}[1]{\newcommand{\@footercatedra}{#1}}

%Indice
%\unsection{},\unsubsection{},etc. Titulos no numerados/listados
\newcommand{\unsection}[1]{\addcontentsline{toc}{section}{#1}\section*{#1}}
\newcommand{\unsubsection}[1]{\addcontentsline{toc}{subsection}{#1}\subsection*{#1}}
\newcommand{\unsubsubsection}[1]{\addcontentsline{toc}{subsubsection}{#1}\subsubsection*{#1}}



%CUSTOM
%\lrah,\llah y \llrah flechas con separcion estandar 0.25cm
%\arc{} arco sobre letra/numero solo en ecuaciones
%\Real{} y \Imag{} operadores reales/imaginario solo en ecuaciones
%\idk ¯\_(ツ)_/¯
%\sumsen{}{}{}{}{}{}{}, \multsen{}{}{} ,\ampli{}{}{} diagrama bloques


%\newcommand{\longdiv}{\smash{\mkern-0.43mu\vstretch{1.31}{\hstretch{.7}{)}}\mkern-5.2mu\vstretch{1.31}{\hstretch{.7}{)}}}}
\newcommand{\lrah}{\hspace{0.25cm} \Longrightarrow \hspace{0.25cm}}
\newcommand{\llah}{\hspace{0.25cm} \Longleftarrow \hspace{0.25cm}}
\newcommand{\llrah}{\hspace{0.25cm} \Longleftrightarrow \hspace{0.25cm}}
\newcommand{\Real}[1]{\mathbb{R}{e}\{#1\}}
\newcommand{\Imag}[1]{\mathbb{I}{m}\{#1\}}
\newcommand{\arc}[1]{{%
    \setbox9=\hbox{#1}%
    \ooalign{\resizebox{\wd9}{\height}{\texttoptiebar{\phantom{A}}}\cr#1}}}
    
\newcommand{\sumsen}[7]{
\node[draw,circle,minimum size=\scal cm,] (#7) at (#1,#2){};
\draw (#7.north east) -- (#7.south west)(#7.north west) -- (#7.south east);
\draw (#7.north east) -- (#7.south west)
(#7.north west) -- (#7.south east);
\node[left=-1pt] at (#7.center){\tiny $#3$};
\node[below] at (#7.center){\tiny $#4$};
\node[right=1pt] at (#7.center){\tiny $#5$};
\node[above] at (#7.center){\tiny $#6$};
}

\newcommand{\multsen}[3]{
\node [shape=signal,draw,signal to=east,  minimum width=1.618*\const*\scal cm,  minimum height=\const*\scal cm] (#3) at (#1,#2) [anchor= 150,draw] {};
\draw [line width=1.2pt] ([shift={(-5pt, -5pt)}] #3.north east) -- ([shift={(5pt, 5pt)}] #3.south west);
\draw [line width=1.2pt] ([shift={(5pt, -5pt)}] #3.north west) -- ([shift={(-5pt, 5pt)}] #3.south east);
}

\newcommand{\ampli}[3]{
\node [isosceles triangle, draw, minimum width=\scal*\const*1.618 cm](#3) at (#1,#2) [anchor= north west,draw] {};
}

\newcommand{\idk}[1][]{%
\begin{tikzpicture}[baseline,x=0.8\ht\strutbox,y=0.8\ht\strutbox,line width=0.125ex,#1]
\def\arm{(-2.5,0.95) to (-2,0.95) (-1.9,1) to (-1.5,0) (-1.35,0) to (-0.8,0)};
\draw \arm;
\draw[xscale=-1] \arm;
\def\headpart{(0.6,0) arc[start angle=-40, end angle=40,x radius=0.6,y radius=0.8]};
\draw \headpart;
\draw[xscale=-1] \headpart;
\def\eye{(-0.075,0.15) .. controls (0.02,0) .. (0.075,-0.15)};
\draw[shift={(-0.3,0.8)}] \eye;
\draw[shift={(0,0.85)}] \eye;
% draw mouth
\draw (-0.1,0.2) to [out=15,in=-100] (0.4,0.95); 
\end{tikzpicture}}
