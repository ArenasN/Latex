%En este segundo trabajo práctico de laboratorio de la materia Medidas Electrónicas I, se analizarán y medirán señales en el dominio del tiempo con osciloscopios tanto analógicos como digitales. En esta oportunidad, se realizarán múltiples experimentos donde se verán de manera práctica los conceptos aprendidos en clase, con el objetivo de adquirir experiencia en el uso de osciloscopios para efectuar el análisis y la medición de algunos parámetros en distintos tipos de formas de ondas.
\section{Introducción}
\subsection{Roles de los Integrantes}

La división de las tareas dentro de nuestro grupo en este trabajo será la siguiente: 

\begin{table}[h!]
    \centering
    \begin{tabular}{|c|c|}
    \hline
        Alumno & Rol \\
    \hline
        Musso, Lucas & Coordinador \\ 
        Arenas, Nicolás & Operador 1 \\
        Palacios, Alexandro & Operador 2 \\
        Robertson, Máximo & Documentación \\
    \hline
        \end{tabular}
        \def\tablename{Tabla} 
        \caption{Tabla de asignación de roles para cada integrante}
        \label{tab:roles}
\end{table}

La fecha de entrega estipulada en el cronograma entregado a los docentes para este trabajo práctico es del \textbf{\textit{25 de abril del 2024}}.

Y la fecha en que el equipo rendirá el coloquio oral será también el día \textbf{\textit{25 de abril del 2024}}.



\subsection{Grilla de Evaluación}

\begin{table}[H]
    \centering
    \scalebox{0.895}{
    \begin{tabular}{|c|p{6.5cm}|c|c|c|c|c|}
    \hline
        \multirow{2}{*}{ID} & \multirow{2}{*}{Criterio de Evaluación} & \multicolumn{4}{c|}{\%Obt.} & \multirow{2}{*}{\%max} \\ 
        \cline{3-6}
        ~ & ~ & ECG & EO1 & EO2 & ED & ~ \\ \hline
        CEval 1 & Identifica los datos necesarios para determinar especificaciones de los instrumentos disponibles & ~ & ~ & ~ & ~ & 5\% \\ \hline
        CEval 8 & Identifica los elementos necesarios para realizar el trabajo requerido & ~ & ~ & ~ & ~ & 5\% \\ \hline
        CEval 10 & Adquiere los conocimientos necesarios para la correcta implementación del procedimiento de medición & ~ & ~ & ~ & ~ & 5\% \\ \hline
        CEval 5 & Trabaja en forma grupal para completar todas las tareas establecidas en el procedimiento indicado para la realización del trabajo practico & ~ & ~ & ~ & ~ & 10\% \\ \hline
        CEval 11 & Realiza los cálculos necesarios para determinar de forma empírica los parámetros de los amplificadores en cada una de las configuraciones planteadas & ~ & ~ & ~ & ~ & 5\% \\ \hline
        CEval 13 & Realiza los cálculos requeridos para determinar los parámetros del amplificador en las configuraciones planteadas de forma analítica & ~ & ~ & ~ & ~ & 5\% \\ \hline
        CEval 14 & Realiza las mediciones para determinar los parámetros del amplificador en las configuraciones planteadas de forma empírica & ~ & ~ & ~ & ~ & 10\% \\ \hline
        CEval 23 & Hace búsqueda y selección de información  relevante para validar las mediciones realizadas & ~ & ~ & ~ & ~ & 5\% \\ \hline
        CEval 4 & Documenta con información precisa el resultado de las mediciones. & ~ & ~ & ~ & ~ & 10\% \\ \hline
        CEval 12 & Evalúa los resultados de las mediciones para determinar la validez de las mismas & ~ & ~ & ~ & ~ & 10\% \\ \hline
        CEval 3 & Realiza el informe técnico con información precisa para dar a conocer los resultados de las mediciones & ~ & ~ & ~ & ~ & 10\% \\ \hline
        CEval 7 & Expone de forma grupal inconvenientes, experiencia generada y conclusiones acerca del trabajo realizado & ~ & ~ & ~ & ~ & 10\% \\ \hline
        CEval 32 & Expone de forma individual inconvenientes, experiencia generada y conclusiones acerca del trabajo realizado & ~ & ~ & ~ & ~ & 10\% \\ \hline
        ~ & TOTAL & ~ & ~ & ~ & ~ & 100\% \\ \hline
    \end{tabular}}
    \def\tablename{Tabla} 
    \caption{Grilla de Evaluación}
\end{table}