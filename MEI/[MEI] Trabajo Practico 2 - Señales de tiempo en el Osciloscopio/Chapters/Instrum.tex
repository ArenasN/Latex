Los instrumentos y materiales utilizados a lo largo de todo este trabajo práctico, son los siguientes:

\begin{itemize}
    \item Osciloscopios Analógicos: marca PINTEK, modelo PS-200; marca LEADER, modelo 8041.
    \item Osciloscopio Digital: marca RIGOL, modelo DS1052E.
    \item 2 Puntas de Osciloscopio Hantek de 100 Mhz.
    \item Generador de funciones: marca GW INSTEK, modelo GFG 3015.
    \item Fuente de tensión variable del laboratorio central de la UTN-FRC.
    \item Kit del TP2 de Medidas Electrónicas (disponible en el laboratorio central de electrónica de la UTN-FRC). Contiene una fuente con salida regulada y no regulada de aprox 12V; un circutio generador de una onda de aprox. 400 Hz; un marcador de teléfono antiguo (disco) con un circuito incorporado para poder ver las señales; y una placa auxiliar para generar un tren de pulsos diferencial.
    
\end{itemize}