En este segundo trabajo práctico de laboratorio de la materia Medidas Electrónicas I, se analizarán y medirán señales en el dominio del tiempo con osciloscopios tanto analógicos como digitales. En esta oportunidad, se realizarán múltiples experimentos donde se verán de manera práctica los conceptos aprendidos en clase, con el objetivo de adquirir experiencia en el uso de osciloscopios para efectuar el análisis y la medición de algunos parámetros en distintos tipos de formas de ondas.

La división de las tareas dentro de nuestro grupo en este trabajo será la siguiente: 

\begin{table}[h!]
    \centering
    \begin{tabular}{|c|c|}
    \hline
        Alumno & Rol \\
    \hline
        Robertson, Máximo & Coordinador \\ 
        Musso, Lucas & Operador 1 \\
        Arenas, Nicolás & Operador 2 \\
        Palacios, Alexandro & Documentación \\
    \hline
        \end{tabular}
        \def\tablename{Tabla} 
        \caption{Tabla de asignación de roles para cada integrante}
        \label{tab:roles}
\end{table}

La fecha de entrega estipulada en el cronograma entregado a los docentes para este trabajo práctico es del \textbf{\textit{25 de abril del 2024}}.

Y la fecha en que el equipo rendirá el coloquio oral será también el día \textbf{\textit{25 de abril del 2024}}.

%Algunos de los instrumentos utilizados fueron:

%\begin{itemize}
   % \item Generador de Funciones GFG
%\end{itemize}

